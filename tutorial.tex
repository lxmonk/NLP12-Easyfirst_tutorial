% Created 2012-08-14 Tue 13:31
\documentclass[11pt]{article}
\usepackage[utf8]{inputenc}
\usepackage[T1]{fontenc}
\usepackage{fixltx2e}
\usepackage{graphicx}
\usepackage{longtable}
\usepackage{float}
\usepackage{wrapfig}
\usepackage{soul}
\usepackage{textcomp}
\usepackage{marvosym}
\usepackage{wasysym}
\usepackage{latexsym}
\usepackage{amssymb}
\usepackage{hyperref}
\tolerance=1000
\providecommand{\alert}[1]{\textbf{#1}}

\title{Tutorial: An Efficient Algorithm for Easy-First Non-Directional Dependency Parsing -- writing the code}
\author{Aviad Reich}
\date{2012-07-21 Sat}
\hypersetup{
  pdfkeywords={},
  pdfsubject={},
  pdfcreator={Emacs Org-mode version 7.8.09}}

\begin{document}

\maketitle

\setcounter{tocdepth}{3}
\tableofcontents
\vspace*{1cm}


\section{How to contribute to this document}
\label{sec-1}

  
This document exists in XHTML, latex, text and org-mode
versions. Editing the first three is straightforward, but will force
all future contributors (if such will exist) to use the same
version/format, since this will break the compatibility of the
different formats. Do this only if you're sure the future contributors
will be OK with it, or you know they will not exist (for example: you
are now completing the document). If you have done so, be sure to
either label the other versions as outdated/irrelevant, or simply
delete them.

Another option is to edit the text file `tutorial.org' in you favorite
text editor. If you ignore lines 5-15, it's pretty much
straightforward to understand the usage: 
The number of left-aligned `*' before a line determine it's `depth' in
the document (more is deeper), and elements are automatically nested in
the immediate higher level item. 

if you want $\TeX{}$, or $\LaTeX$,
include its code inside adjacent dollar signs: \\
\texttt{\$[latex code here]\$}.
Just look at `tutorial.org' for examples.

To use the other many wonderful, yet simple to use, features of
org-mode, like auto-numbering of items, footnotes and others, 
it's recommended you read the \href{http://orgmode.org/}{orgmode website} and the (relevant) \href{http://orgmode.org/org-mode-documentation.html}{docs},
or the tutorial that comes with it as part of Emacs, or \href{http://orgmode.org/worg/org-tutorials/}{these tutorials}.

\begin{description}
\item[Exporting] To export to HTML, tex and text once you're done, open
               Emacs (you  know where to get it), press Ctrl-x Ctrl-f
               and type the path to the file. Then, press Ctrl-c
               Ctrl-e h (for HTML), Ctrl-c Ctrl-e p (for pdf), Ctrl-c
               Ctrl-e l (for latex), Ctrl-c Ctrl-e u (for unicode
               text, a for ascii). Many other export format exist -
               you'll find it in the ``Emacs-style popup''
               window. \\
               \textbf{MAKE SURE YOU EXPORT IN ALL FORMATS ONCE                YOU'RE DONE, SO COMPATIBILITY IS KEPT.}
\end{description}

As this is written (August 2012), I care about this document, and
would be happy to extend my help if it's wanted. To email me use the
first 3 letters of `Aviad', followed by a dot (`.') and the
first 3 of `Reich'. Then mail me at: \texttt{[what-you-got]@gmail.com}.

Additionally, there is a github repo:
\href{https://github.com/lxmonk/NLP12-Easyfirst_tutorial}{https://github.com/lxmonk/NLP12-Easyfirst\_tutorial}, that you can clone
or fork. If you do, and you've created a new one - change this
address. Otherwise, let me know and I'll update.
\section{An intro to dep parsing}
\label{sec-2}
\subsection{Formats of the data}
\label{sec-2-1}
\subsection{Example sentences}
\label{sec-2-2}
\subsection{Rendering of trees in text and graphically}
\label{sec-2-3}
\section{Intro to dep parsing using transition based parsing}
\label{sec-3}
\subsection{How shift reduce works}
\label{sec-3-1}
\subsection{Turning a tree into a sequence of shift reduce transitions}
\label{sec-3-2}
\section{Malt like parsing}
\label{sec-4}
\subsection{Training a classifier to learn which transition is best at each step}
\label{sec-4-1}
\subsection{Typical features used for malt}
\label{sec-4-2}
\section{Evaluation methods for dep parsing}
\label{sec-5}
\section{Evaluation of our malt parser}
\label{sec-6}
\section{Easy First}
\label{sec-7}
\subsection{Read the paper}
\label{sec-7-1}


The article: \\
\textbf{Easy First Dependency Parsing of Modern Hebrew}, \\
   Yoav Goldberg and Michael Elhadad, \\
   \emph{SPMRL 2010 (NAACL Workshop on Statistical Parsing of    Morphologically-rich Languages)}


It can be obtained from \href{http://www.cs.bgu.ac.il/~yoavg/publications/naacl2010dep.pdf}{Yoav Goldberg's BGU webpage}, or at the acm
website: 
\href{http://dl.acm.org/citation.cfm?id=1857999.1858114}{http://dl.acm.org/citation.cfm?id=1857999.1858114}.


   
\subsection{Quiz on the paper}
\label{sec-7-2}
\subsubsection{Perceptron Classifier}
\label{sec-7-2-1}
\subsubsection{Cython}
\label{sec-7-2-2}

    
\section{Cython primer}
\label{sec-8}

\end{document}